\documentclass[11pt]{article}

    \usepackage[breakable]{tcolorbox}
    \usepackage{parskip} % Stop auto-indenting (to mimic markdown behaviour)
    \usepackage[T2A]{fontenc}
    \usepackage[utf8]{inputenc}
    \usepackage[english, russian]{babel}
    \usepackage{mathtext}
    \usepackage{iftex}

    % Basic figure setup, for now with no caption control since it's done
    % automatically by Pandoc (which extracts ![](path) syntax from Markdown).
    \usepackage{graphicx}
    % Maintain compatibility with old templates. Remove in nbconvert 6.0
    \let\Oldincludegraphics\includegraphics
    % Ensure that by default, figures have no caption (until we provide a
    % proper Figure object with a Caption API and a way to capture that
    % in the conversion process - todo).
    \usepackage{caption}
    \DeclareCaptionFormat{nocaption}{}
    \captionsetup{format=nocaption,aboveskip=0pt,belowskip=0pt}

    \usepackage[Export]{adjustbox} % Used to constrain images to a maximum size
    \adjustboxset{max size={0.9\linewidth}{0.9\paperheight}}
    \usepackage{float}
    \floatplacement{figure}{H} % forces figures to be placed at the correct location
    \usepackage{xcolor} % Allow colors to be defined
    \usepackage{enumerate} % Needed for markdown enumerations to work
    \usepackage{geometry} % Used to adjust the document margins
    \usepackage{amsmath} % Equations
    \usepackage{amssymb} % Equations
    \usepackage{textcomp} % defines textquotesingle
    % Hack from http://tex.stackexchange.com/a/47451/13684:
    \AtBeginDocument{%
        \def\PYZsq{\textquotesingle}% Upright quotes in Pygmentized code
    }
    \usepackage{upquote} % Upright quotes for verbatim code
    \usepackage{eurosym} % defines \euro
    \usepackage[mathletters]{ucs} % Extended unicode (utf-8) support
    \usepackage{fancyvrb} % verbatim replacement that allows latex
    \usepackage{grffile} % extends the file name processing of package graphics 
                         % to support a larger range
    \makeatletter % fix for grffile with XeLaTeX
    \def\Gread@@xetex#1{%
      \IfFileExists{"\Gin@base".bb}%
      {\Gread@eps{\Gin@base.bb}}%
      {\Gread@@xetex@aux#1}%
    }
    \makeatother

    % The hyperref package gives us a pdf with properly built
    % internal navigation ('pdf bookmarks' for the table of contents,
    % internal cross-reference links, web links for URLs, etc.)
    \usepackage{hyperref}
    % The default LaTeX title has an obnoxious amount of whitespace. By default,
    % titling removes some of it. It also provides customization options.
    \usepackage{titling}
    \usepackage{longtable} % longtable support required by pandoc >1.10
    \usepackage{booktabs}  % table support for pandoc > 1.12.2
    \usepackage[inline]{enumitem} % IRkernel/repr support (it uses the enumerate* environment)
    \usepackage[normalem]{ulem} % ulem is needed to support strikethroughs (\sout)
                                % normalem makes italics be italics, not underlines
    \usepackage{mathrsfs}
    

    
    % Colors for the hyperref package
    \definecolor{urlcolor}{rgb}{0,.145,.698}
    \definecolor{linkcolor}{rgb}{.71,0.21,0.01}
    \definecolor{citecolor}{rgb}{.12,.54,.11}

    % ANSI colors
    \definecolor{ansi-black}{HTML}{3E424D}
    \definecolor{ansi-black-intense}{HTML}{282C36}
    \definecolor{ansi-red}{HTML}{E75C58}
    \definecolor{ansi-red-intense}{HTML}{B22B31}
    \definecolor{ansi-green}{HTML}{00A250}
    \definecolor{ansi-green-intense}{HTML}{007427}
    \definecolor{ansi-yellow}{HTML}{DDB62B}
    \definecolor{ansi-yellow-intense}{HTML}{B27D12}
    \definecolor{ansi-blue}{HTML}{208FFB}
    \definecolor{ansi-blue-intense}{HTML}{0065CA}
    \definecolor{ansi-magenta}{HTML}{D160C4}
    \definecolor{ansi-magenta-intense}{HTML}{A03196}
    \definecolor{ansi-cyan}{HTML}{60C6C8}
    \definecolor{ansi-cyan-intense}{HTML}{258F8F}
    \definecolor{ansi-white}{HTML}{C5C1B4}
    \definecolor{ansi-white-intense}{HTML}{A1A6B2}
    \definecolor{ansi-default-inverse-fg}{HTML}{FFFFFF}
    \definecolor{ansi-default-inverse-bg}{HTML}{000000}

    % commands and environments needed by pandoc snippets
    % extracted from the output of `pandoc -s`
    \providecommand{\tightlist}{%
      \setlength{\itemsep}{0pt}\setlength{\parskip}{0pt}}
    \DefineVerbatimEnvironment{Highlighting}{Verbatim}{commandchars=\\\{\}}
    % Add ',fontsize=\small' for more characters per line
    \newenvironment{Shaded}{}{}
    \newcommand{\KeywordTok}[1]{\textcolor[rgb]{0.00,0.44,0.13}{\textbf{{#1}}}}
    \newcommand{\DataTypeTok}[1]{\textcolor[rgb]{0.56,0.13,0.00}{{#1}}}
    \newcommand{\DecValTok}[1]{\textcolor[rgb]{0.25,0.63,0.44}{{#1}}}
    \newcommand{\BaseNTok}[1]{\textcolor[rgb]{0.25,0.63,0.44}{{#1}}}
    \newcommand{\FloatTok}[1]{\textcolor[rgb]{0.25,0.63,0.44}{{#1}}}
    \newcommand{\CharTok}[1]{\textcolor[rgb]{0.25,0.44,0.63}{{#1}}}
    \newcommand{\StringTok}[1]{\textcolor[rgb]{0.25,0.44,0.63}{{#1}}}
    \newcommand{\CommentTok}[1]{\textcolor[rgb]{0.38,0.63,0.69}{\textit{{#1}}}}
    \newcommand{\OtherTok}[1]{\textcolor[rgb]{0.00,0.44,0.13}{{#1}}}
    \newcommand{\AlertTok}[1]{\textcolor[rgb]{1.00,0.00,0.00}{\textbf{{#1}}}}
    \newcommand{\FunctionTok}[1]{\textcolor[rgb]{0.02,0.16,0.49}{{#1}}}
    \newcommand{\RegionMarkerTok}[1]{{#1}}
    \newcommand{\ErrorTok}[1]{\textcolor[rgb]{1.00,0.00,0.00}{\textbf{{#1}}}}
    \newcommand{\NormalTok}[1]{{#1}}
    
    % Additional commands for more recent versions of Pandoc
    \newcommand{\ConstantTok}[1]{\textcolor[rgb]{0.53,0.00,0.00}{{#1}}}
    \newcommand{\SpecialCharTok}[1]{\textcolor[rgb]{0.25,0.44,0.63}{{#1}}}
    \newcommand{\VerbatimStringTok}[1]{\textcolor[rgb]{0.25,0.44,0.63}{{#1}}}
    \newcommand{\SpecialStringTok}[1]{\textcolor[rgb]{0.73,0.40,0.53}{{#1}}}
    \newcommand{\ImportTok}[1]{{#1}}
    \newcommand{\DocumentationTok}[1]{\textcolor[rgb]{0.73,0.13,0.13}{\textit{{#1}}}}
    \newcommand{\AnnotationTok}[1]{\textcolor[rgb]{0.38,0.63,0.69}{\textbf{\textit{{#1}}}}}
    \newcommand{\CommentVarTok}[1]{\textcolor[rgb]{0.38,0.63,0.69}{\textbf{\textit{{#1}}}}}
    \newcommand{\VariableTok}[1]{\textcolor[rgb]{0.10,0.09,0.49}{{#1}}}
    \newcommand{\ControlFlowTok}[1]{\textcolor[rgb]{0.00,0.44,0.13}{\textbf{{#1}}}}
    \newcommand{\OperatorTok}[1]{\textcolor[rgb]{0.40,0.40,0.40}{{#1}}}
    \newcommand{\BuiltInTok}[1]{{#1}}
    \newcommand{\ExtensionTok}[1]{{#1}}
    \newcommand{\PreprocessorTok}[1]{\textcolor[rgb]{0.74,0.48,0.00}{{#1}}}
    \newcommand{\AttributeTok}[1]{\textcolor[rgb]{0.49,0.56,0.16}{{#1}}}
    \newcommand{\InformationTok}[1]{\textcolor[rgb]{0.38,0.63,0.69}{\textbf{\textit{{#1}}}}}
    \newcommand{\WarningTok}[1]{\textcolor[rgb]{0.38,0.63,0.69}{\textbf{\textit{{#1}}}}}
    
    
    % Define a nice break command that doesn't care if a line doesn't already
    % exist.
    \def\br{\hspace*{\fill} \\* }
    % Math Jax compatibility definitions
    \def\gt{>}
    \def\lt{<}
    \let\Oldtex\TeX
    \let\Oldlatex\LaTeX
    \renewcommand{\TeX}{\textrm{\Oldtex}}
    \renewcommand{\LaTeX}{\textrm{\Oldlatex}}
    % Document parameters
    % Document title
    \title{main}
    
    
    
    
    
% Pygments definitions
\makeatletter
\def\PY@reset{\let\PY@it=\relax \let\PY@bf=\relax%
    \let\PY@ul=\relax \let\PY@tc=\relax%
    \let\PY@bc=\relax \let\PY@ff=\relax}
\def\PY@tok#1{\csname PY@tok@#1\endcsname}
\def\PY@toks#1+{\ifx\relax#1\empty\else%
    \PY@tok{#1}\expandafter\PY@toks\fi}
\def\PY@do#1{\PY@bc{\PY@tc{\PY@ul{%
    \PY@it{\PY@bf{\PY@ff{#1}}}}}}}
\def\PY#1#2{\PY@reset\PY@toks#1+\relax+\PY@do{#2}}

\expandafter\def\csname PY@tok@w\endcsname{\def\PY@tc##1{\textcolor[rgb]{0.73,0.73,0.73}{##1}}}
\expandafter\def\csname PY@tok@c\endcsname{\let\PY@it=\textit\def\PY@tc##1{\textcolor[rgb]{0.25,0.50,0.50}{##1}}}
\expandafter\def\csname PY@tok@cp\endcsname{\def\PY@tc##1{\textcolor[rgb]{0.74,0.48,0.00}{##1}}}
\expandafter\def\csname PY@tok@k\endcsname{\let\PY@bf=\textbf\def\PY@tc##1{\textcolor[rgb]{0.00,0.50,0.00}{##1}}}
\expandafter\def\csname PY@tok@kp\endcsname{\def\PY@tc##1{\textcolor[rgb]{0.00,0.50,0.00}{##1}}}
\expandafter\def\csname PY@tok@kt\endcsname{\def\PY@tc##1{\textcolor[rgb]{0.69,0.00,0.25}{##1}}}
\expandafter\def\csname PY@tok@o\endcsname{\def\PY@tc##1{\textcolor[rgb]{0.40,0.40,0.40}{##1}}}
\expandafter\def\csname PY@tok@ow\endcsname{\let\PY@bf=\textbf\def\PY@tc##1{\textcolor[rgb]{0.67,0.13,1.00}{##1}}}
\expandafter\def\csname PY@tok@nb\endcsname{\def\PY@tc##1{\textcolor[rgb]{0.00,0.50,0.00}{##1}}}
\expandafter\def\csname PY@tok@nf\endcsname{\def\PY@tc##1{\textcolor[rgb]{0.00,0.00,1.00}{##1}}}
\expandafter\def\csname PY@tok@nc\endcsname{\let\PY@bf=\textbf\def\PY@tc##1{\textcolor[rgb]{0.00,0.00,1.00}{##1}}}
\expandafter\def\csname PY@tok@nn\endcsname{\let\PY@bf=\textbf\def\PY@tc##1{\textcolor[rgb]{0.00,0.00,1.00}{##1}}}
\expandafter\def\csname PY@tok@ne\endcsname{\let\PY@bf=\textbf\def\PY@tc##1{\textcolor[rgb]{0.82,0.25,0.23}{##1}}}
\expandafter\def\csname PY@tok@nv\endcsname{\def\PY@tc##1{\textcolor[rgb]{0.10,0.09,0.49}{##1}}}
\expandafter\def\csname PY@tok@no\endcsname{\def\PY@tc##1{\textcolor[rgb]{0.53,0.00,0.00}{##1}}}
\expandafter\def\csname PY@tok@nl\endcsname{\def\PY@tc##1{\textcolor[rgb]{0.63,0.63,0.00}{##1}}}
\expandafter\def\csname PY@tok@ni\endcsname{\let\PY@bf=\textbf\def\PY@tc##1{\textcolor[rgb]{0.60,0.60,0.60}{##1}}}
\expandafter\def\csname PY@tok@na\endcsname{\def\PY@tc##1{\textcolor[rgb]{0.49,0.56,0.16}{##1}}}
\expandafter\def\csname PY@tok@nt\endcsname{\let\PY@bf=\textbf\def\PY@tc##1{\textcolor[rgb]{0.00,0.50,0.00}{##1}}}
\expandafter\def\csname PY@tok@nd\endcsname{\def\PY@tc##1{\textcolor[rgb]{0.67,0.13,1.00}{##1}}}
\expandafter\def\csname PY@tok@s\endcsname{\def\PY@tc##1{\textcolor[rgb]{0.73,0.13,0.13}{##1}}}
\expandafter\def\csname PY@tok@sd\endcsname{\let\PY@it=\textit\def\PY@tc##1{\textcolor[rgb]{0.73,0.13,0.13}{##1}}}
\expandafter\def\csname PY@tok@si\endcsname{\let\PY@bf=\textbf\def\PY@tc##1{\textcolor[rgb]{0.73,0.40,0.53}{##1}}}
\expandafter\def\csname PY@tok@se\endcsname{\let\PY@bf=\textbf\def\PY@tc##1{\textcolor[rgb]{0.73,0.40,0.13}{##1}}}
\expandafter\def\csname PY@tok@sr\endcsname{\def\PY@tc##1{\textcolor[rgb]{0.73,0.40,0.53}{##1}}}
\expandafter\def\csname PY@tok@ss\endcsname{\def\PY@tc##1{\textcolor[rgb]{0.10,0.09,0.49}{##1}}}
\expandafter\def\csname PY@tok@sx\endcsname{\def\PY@tc##1{\textcolor[rgb]{0.00,0.50,0.00}{##1}}}
\expandafter\def\csname PY@tok@m\endcsname{\def\PY@tc##1{\textcolor[rgb]{0.40,0.40,0.40}{##1}}}
\expandafter\def\csname PY@tok@gh\endcsname{\let\PY@bf=\textbf\def\PY@tc##1{\textcolor[rgb]{0.00,0.00,0.50}{##1}}}
\expandafter\def\csname PY@tok@gu\endcsname{\let\PY@bf=\textbf\def\PY@tc##1{\textcolor[rgb]{0.50,0.00,0.50}{##1}}}
\expandafter\def\csname PY@tok@gd\endcsname{\def\PY@tc##1{\textcolor[rgb]{0.63,0.00,0.00}{##1}}}
\expandafter\def\csname PY@tok@gi\endcsname{\def\PY@tc##1{\textcolor[rgb]{0.00,0.63,0.00}{##1}}}
\expandafter\def\csname PY@tok@gr\endcsname{\def\PY@tc##1{\textcolor[rgb]{1.00,0.00,0.00}{##1}}}
\expandafter\def\csname PY@tok@ge\endcsname{\let\PY@it=\textit}
\expandafter\def\csname PY@tok@gs\endcsname{\let\PY@bf=\textbf}
\expandafter\def\csname PY@tok@gp\endcsname{\let\PY@bf=\textbf\def\PY@tc##1{\textcolor[rgb]{0.00,0.00,0.50}{##1}}}
\expandafter\def\csname PY@tok@go\endcsname{\def\PY@tc##1{\textcolor[rgb]{0.53,0.53,0.53}{##1}}}
\expandafter\def\csname PY@tok@gt\endcsname{\def\PY@tc##1{\textcolor[rgb]{0.00,0.27,0.87}{##1}}}
\expandafter\def\csname PY@tok@err\endcsname{\def\PY@bc##1{\setlength{\fboxsep}{0pt}\fcolorbox[rgb]{1.00,0.00,0.00}{1,1,1}{\strut ##1}}}
\expandafter\def\csname PY@tok@kc\endcsname{\let\PY@bf=\textbf\def\PY@tc##1{\textcolor[rgb]{0.00,0.50,0.00}{##1}}}
\expandafter\def\csname PY@tok@kd\endcsname{\let\PY@bf=\textbf\def\PY@tc##1{\textcolor[rgb]{0.00,0.50,0.00}{##1}}}
\expandafter\def\csname PY@tok@kn\endcsname{\let\PY@bf=\textbf\def\PY@tc##1{\textcolor[rgb]{0.00,0.50,0.00}{##1}}}
\expandafter\def\csname PY@tok@kr\endcsname{\let\PY@bf=\textbf\def\PY@tc##1{\textcolor[rgb]{0.00,0.50,0.00}{##1}}}
\expandafter\def\csname PY@tok@bp\endcsname{\def\PY@tc##1{\textcolor[rgb]{0.00,0.50,0.00}{##1}}}
\expandafter\def\csname PY@tok@fm\endcsname{\def\PY@tc##1{\textcolor[rgb]{0.00,0.00,1.00}{##1}}}
\expandafter\def\csname PY@tok@vc\endcsname{\def\PY@tc##1{\textcolor[rgb]{0.10,0.09,0.49}{##1}}}
\expandafter\def\csname PY@tok@vg\endcsname{\def\PY@tc##1{\textcolor[rgb]{0.10,0.09,0.49}{##1}}}
\expandafter\def\csname PY@tok@vi\endcsname{\def\PY@tc##1{\textcolor[rgb]{0.10,0.09,0.49}{##1}}}
\expandafter\def\csname PY@tok@vm\endcsname{\def\PY@tc##1{\textcolor[rgb]{0.10,0.09,0.49}{##1}}}
\expandafter\def\csname PY@tok@sa\endcsname{\def\PY@tc##1{\textcolor[rgb]{0.73,0.13,0.13}{##1}}}
\expandafter\def\csname PY@tok@sb\endcsname{\def\PY@tc##1{\textcolor[rgb]{0.73,0.13,0.13}{##1}}}
\expandafter\def\csname PY@tok@sc\endcsname{\def\PY@tc##1{\textcolor[rgb]{0.73,0.13,0.13}{##1}}}
\expandafter\def\csname PY@tok@dl\endcsname{\def\PY@tc##1{\textcolor[rgb]{0.73,0.13,0.13}{##1}}}
\expandafter\def\csname PY@tok@s2\endcsname{\def\PY@tc##1{\textcolor[rgb]{0.73,0.13,0.13}{##1}}}
\expandafter\def\csname PY@tok@sh\endcsname{\def\PY@tc##1{\textcolor[rgb]{0.73,0.13,0.13}{##1}}}
\expandafter\def\csname PY@tok@s1\endcsname{\def\PY@tc##1{\textcolor[rgb]{0.73,0.13,0.13}{##1}}}
\expandafter\def\csname PY@tok@mb\endcsname{\def\PY@tc##1{\textcolor[rgb]{0.40,0.40,0.40}{##1}}}
\expandafter\def\csname PY@tok@mf\endcsname{\def\PY@tc##1{\textcolor[rgb]{0.40,0.40,0.40}{##1}}}
\expandafter\def\csname PY@tok@mh\endcsname{\def\PY@tc##1{\textcolor[rgb]{0.40,0.40,0.40}{##1}}}
\expandafter\def\csname PY@tok@mi\endcsname{\def\PY@tc##1{\textcolor[rgb]{0.40,0.40,0.40}{##1}}}
\expandafter\def\csname PY@tok@il\endcsname{\def\PY@tc##1{\textcolor[rgb]{0.40,0.40,0.40}{##1}}}
\expandafter\def\csname PY@tok@mo\endcsname{\def\PY@tc##1{\textcolor[rgb]{0.40,0.40,0.40}{##1}}}
\expandafter\def\csname PY@tok@ch\endcsname{\let\PY@it=\textit\def\PY@tc##1{\textcolor[rgb]{0.25,0.50,0.50}{##1}}}
\expandafter\def\csname PY@tok@cm\endcsname{\let\PY@it=\textit\def\PY@tc##1{\textcolor[rgb]{0.25,0.50,0.50}{##1}}}
\expandafter\def\csname PY@tok@cpf\endcsname{\let\PY@it=\textit\def\PY@tc##1{\textcolor[rgb]{0.25,0.50,0.50}{##1}}}
\expandafter\def\csname PY@tok@c1\endcsname{\let\PY@it=\textit\def\PY@tc##1{\textcolor[rgb]{0.25,0.50,0.50}{##1}}}
\expandafter\def\csname PY@tok@cs\endcsname{\let\PY@it=\textit\def\PY@tc##1{\textcolor[rgb]{0.25,0.50,0.50}{##1}}}

\def\PYZbs{\char`\\}
\def\PYZus{\char`\_}
\def\PYZob{\char`\{}
\def\PYZcb{\char`\}}
\def\PYZca{\char`\^}
\def\PYZam{\char`\&}
\def\PYZlt{\char`\<}
\def\PYZgt{\char`\>}
\def\PYZsh{\char`\#}
\def\PYZpc{\char`\%}
\def\PYZdl{\char`\$}
\def\PYZhy{\char`\-}
\def\PYZsq{\char`\'}
\def\PYZdq{\char`\"}
\def\PYZti{\char`\~}
% for compatibility with earlier versions
\def\PYZat{@}
\def\PYZlb{[}
\def\PYZrb{]}
\makeatother


    % For linebreaks inside Verbatim environment from package fancyvrb. 
    \makeatletter
        \newbox\Wrappedcontinuationbox 
        \newbox\Wrappedvisiblespacebox 
        \newcommand*\Wrappedvisiblespace {\textcolor{red}{\textvisiblespace}} 
        \newcommand*\Wrappedcontinuationsymbol {\textcolor{red}{\llap{\tiny$\m@th\hookrightarrow$}}} 
        \newcommand*\Wrappedcontinuationindent {3ex } 
        \newcommand*\Wrappedafterbreak {\kern\Wrappedcontinuationindent\copy\Wrappedcontinuationbox} 
        % Take advantage of the already applied Pygments mark-up to insert 
        % potential linebreaks for TeX processing. 
        %        {, <, #, %, $, ' and ": go to next line. 
        %        _, }, ^, &, >, - and ~: stay at end of broken line. 
        % Use of \textquotesingle for straight quote. 
        \newcommand*\Wrappedbreaksatspecials {% 
            \def\PYGZus{\discretionary{\char`\_}{\Wrappedafterbreak}{\char`\_}}% 
            \def\PYGZob{\discretionary{}{\Wrappedafterbreak\char`\{}{\char`\{}}% 
            \def\PYGZcb{\discretionary{\char`\}}{\Wrappedafterbreak}{\char`\}}}% 
            \def\PYGZca{\discretionary{\char`\^}{\Wrappedafterbreak}{\char`\^}}% 
            \def\PYGZam{\discretionary{\char`\&}{\Wrappedafterbreak}{\char`\&}}% 
            \def\PYGZlt{\discretionary{}{\Wrappedafterbreak\char`\<}{\char`\<}}% 
            \def\PYGZgt{\discretionary{\char`\>}{\Wrappedafterbreak}{\char`\>}}% 
            \def\PYGZsh{\discretionary{}{\Wrappedafterbreak\char`\#}{\char`\#}}% 
            \def\PYGZpc{\discretionary{}{\Wrappedafterbreak\char`\%}{\char`\%}}% 
            \def\PYGZdl{\discretionary{}{\Wrappedafterbreak\char`\$}{\char`\$}}% 
            \def\PYGZhy{\discretionary{\char`\-}{\Wrappedafterbreak}{\char`\-}}% 
            \def\PYGZsq{\discretionary{}{\Wrappedafterbreak\textquotesingle}{\textquotesingle}}% 
            \def\PYGZdq{\discretionary{}{\Wrappedafterbreak\char`\"}{\char`\"}}% 
            \def\PYGZti{\discretionary{\char`\~}{\Wrappedafterbreak}{\char`\~}}% 
        } 
        % Some characters . , ; ? ! / are not pygmentized. 
        % This macro makes them "active" and they will insert potential linebreaks 
        \newcommand*\Wrappedbreaksatpunct {% 
            \lccode`\~`\.\lowercase{\def~}{\discretionary{\hbox{\char`\.}}{\Wrappedafterbreak}{\hbox{\char`\.}}}% 
            \lccode`\~`\,\lowercase{\def~}{\discretionary{\hbox{\char`\,}}{\Wrappedafterbreak}{\hbox{\char`\,}}}% 
            \lccode`\~`\;\lowercase{\def~}{\discretionary{\hbox{\char`\;}}{\Wrappedafterbreak}{\hbox{\char`\;}}}% 
            \lccode`\~`\:\lowercase{\def~}{\discretionary{\hbox{\char`\:}}{\Wrappedafterbreak}{\hbox{\char`\:}}}% 
            \lccode`\~`\?\lowercase{\def~}{\discretionary{\hbox{\char`\?}}{\Wrappedafterbreak}{\hbox{\char`\?}}}% 
            \lccode`\~`\!\lowercase{\def~}{\discretionary{\hbox{\char`\!}}{\Wrappedafterbreak}{\hbox{\char`\!}}}% 
            \lccode`\~`\/\lowercase{\def~}{\discretionary{\hbox{\char`\/}}{\Wrappedafterbreak}{\hbox{\char`\/}}}% 
            \catcode`\.\active
            \catcode`\,\active 
            \catcode`\;\active
            \catcode`\:\active
            \catcode`\?\active
            \catcode`\!\active
            \catcode`\/\active 
            \lccode`\~`\~ 	
        }
    \makeatother

    \let\OriginalVerbatim=\Verbatim
    \makeatletter
    \renewcommand{\Verbatim}[1][1]{%
        %\parskip\z@skip
        \sbox\Wrappedcontinuationbox {\Wrappedcontinuationsymbol}%
        \sbox\Wrappedvisiblespacebox {\FV@SetupFont\Wrappedvisiblespace}%
        \def\FancyVerbFormatLine ##1{\hsize\linewidth
            \vtop{\raggedright\hyphenpenalty\z@\exhyphenpenalty\z@
                \doublehyphendemerits\z@\finalhyphendemerits\z@
                \strut ##1\strut}%
        }%
        % If the linebreak is at a space, the latter will be displayed as visible
        % space at end of first line, and a continuation symbol starts next line.
        % Stretch/shrink are however usually zero for typewriter font.
        \def\FV@Space {%
            \nobreak\hskip\z@ plus\fontdimen3\font minus\fontdimen4\font
            \discretionary{\copy\Wrappedvisiblespacebox}{\Wrappedafterbreak}
            {\kern\fontdimen2\font}%
        }%
        
        % Allow breaks at special characters using \PYG... macros.
        \Wrappedbreaksatspecials
        % Breaks at punctuation characters . , ; ? ! and / need catcode=\active 	
        \OriginalVerbatim[#1,codes*=\Wrappedbreaksatpunct]%
    }
    \makeatother

    % Exact colors from NB
    \definecolor{incolor}{HTML}{303F9F}
    \definecolor{outcolor}{HTML}{D84315}
    \definecolor{cellborder}{HTML}{CFCFCF}
    \definecolor{cellbackground}{HTML}{F7F7F7}
    
    % prompt
    \makeatletter
    \newcommand{\boxspacing}{\kern\kvtcb@left@rule\kern\kvtcb@boxsep}
    \makeatother
    \newcommand{\prompt}[4]{
        \ttfamily\llap{{\color{#2}[#3]:\hspace{3pt}#4}}\vspace{-\baselineskip}
    }
    

    
    % Prevent overflowing lines due to hard-to-break entities
    \sloppy 
    % Setup hyperref package
    \hypersetup{
      breaklinks=true,  % so long urls are correctly broken across lines
      colorlinks=true,
      urlcolor=urlcolor,
      linkcolor=linkcolor,
      citecolor=citecolor,
      }
    % Slightly bigger margins than the latex defaults
    
    \geometry{verbose,tmargin=1in,bmargin=1in,lmargin=1in,rmargin=1in}
    
    

\begin{document}
    
    
          \begin{titlepage}
\newpage

\begin{center}
Министерство образования и науки Российской Федерации
ФГАОУ ВО «Северо-Восточный Федеральный Университет имени М.К.Аммосовa» \\
\end{center}


\vspace{25em}

\begin{center}
\textsc{\textbf{Лабораторная работа}} \\
\textsc{\textbf{Моделирование задачи теплопроводности}}
\end{center}

\vspace{20em}



\newbox{\lbox}
\savebox{\lbox}{\hbox{Пупкин Иван Иванович}}
\newlength{\maxl}
\setlength{\maxl}{\wd\lbox}
\hfill\parbox{11cm}{
\hspace*{5cm}\hbox{Студент}:\hfill\hbox to\maxl{Захаров Тимур Захарович\hfill}\\
\hspace*{5cm}Группа:\hfill\hbox to\maxl{ПМИ-20}\\
\hspace*{5cm}Вариант:\hfill\hbox to\maxl{1}\\
}


\vspace{\fill}

\begin{center}
Якутск \\2024
\end{center}

\end{titlepage}  
\Large\textbf{Поставновка задачи}  \\
\textbf{Математическая модель:}  Прямоугольная пластина с изотропным ма- териалом и граничными условиями Дирихле. \\
$$\frac{\partial T}{\partial t}=\nabla\cdot(\lambda\nabla{T})$$
\textbf{Граничные условия:} \\
$$T=0\quad на\;всех\;границах$$
\textbf{Начальные условия:} \\
$$T_0=20$$
\textbf{Коэффициенты:}
$$\lambda=1.0$$
\textbf{Расчётная сетка}
    
    \begin{tcolorbox}[breakable, size=fbox, boxrule=1pt, pad at break*=1mm,colback=cellbackground, colframe=cellborder]
\prompt{In}{incolor}{44}{\boxspacing}
\begin{Verbatim}[commandchars=\\\{\}]
\PY{k+kn}{import} \PY{n+nn}{gmsh}
\end{Verbatim}
\end{tcolorbox}

    \begin{tcolorbox}[breakable, size=fbox, boxrule=1pt, pad at break*=1mm,colback=cellbackground, colframe=cellborder]
\prompt{In}{incolor}{45}{\boxspacing}
\begin{Verbatim}[commandchars=\\\{\}]
\PY{k}{def} \PY{n+nf}{create\PYZus{}model}\PY{p}{(}\PY{n}{width}\PY{p}{,} \PY{n}{height}\PY{p}{,} \PY{n}{p\PYZus{}reactangle}\PY{p}{,} \PY{n}{name}\PY{p}{)}\PY{p}{:}
    \PY{n}{gmsh}\PY{o}{.}\PY{n}{initialize}\PY{p}{(}\PY{p}{)}

    \PY{n}{p1} \PY{o}{=} \PY{n}{gmsh}\PY{o}{.}\PY{n}{model}\PY{o}{.}\PY{n}{geo}\PY{o}{.}\PY{n}{addPoint}\PY{p}{(}\PY{l+m+mi}{0}\PY{p}{,} \PY{l+m+mi}{0}\PY{p}{,} \PY{l+m+mi}{0}\PY{p}{,} \PY{n}{p\PYZus{}reactangle}\PY{p}{)}
    \PY{n}{p2} \PY{o}{=} \PY{n}{gmsh}\PY{o}{.}\PY{n}{model}\PY{o}{.}\PY{n}{geo}\PY{o}{.}\PY{n}{addPoint}\PY{p}{(}\PY{n}{width}\PY{p}{,} \PY{l+m+mi}{0}\PY{p}{,} \PY{l+m+mi}{0}\PY{p}{,} \PY{n}{p\PYZus{}reactangle}\PY{p}{)}
    \PY{n}{p3} \PY{o}{=} \PY{n}{gmsh}\PY{o}{.}\PY{n}{model}\PY{o}{.}\PY{n}{geo}\PY{o}{.}\PY{n}{addPoint}\PY{p}{(}\PY{n}{width}\PY{p}{,} \PY{n}{height}\PY{p}{,} \PY{l+m+mi}{0}\PY{p}{,} \PY{n}{p\PYZus{}reactangle}\PY{p}{)}
    \PY{n}{p4} \PY{o}{=} \PY{n}{gmsh}\PY{o}{.}\PY{n}{model}\PY{o}{.}\PY{n}{geo}\PY{o}{.}\PY{n}{addPoint}\PY{p}{(}\PY{l+m+mi}{0}\PY{p}{,} \PY{n}{height}\PY{p}{,} \PY{l+m+mi}{0}\PY{p}{,} \PY{n}{p\PYZus{}reactangle}\PY{p}{)}

    \PY{c+c1}{\PYZsh{} Создаем линии прямоугольника}
    \PY{n}{l1} \PY{o}{=} \PY{n}{gmsh}\PY{o}{.}\PY{n}{model}\PY{o}{.}\PY{n}{geo}\PY{o}{.}\PY{n}{addLine}\PY{p}{(}\PY{n}{p1}\PY{p}{,} \PY{n}{p2}\PY{p}{)}
    \PY{n}{l2} \PY{o}{=} \PY{n}{gmsh}\PY{o}{.}\PY{n}{model}\PY{o}{.}\PY{n}{geo}\PY{o}{.}\PY{n}{addLine}\PY{p}{(}\PY{n}{p2}\PY{p}{,} \PY{n}{p3}\PY{p}{)}
    \PY{n}{l3} \PY{o}{=} \PY{n}{gmsh}\PY{o}{.}\PY{n}{model}\PY{o}{.}\PY{n}{geo}\PY{o}{.}\PY{n}{addLine}\PY{p}{(}\PY{n}{p3}\PY{p}{,} \PY{n}{p4}\PY{p}{)}
    \PY{n}{l4} \PY{o}{=} \PY{n}{gmsh}\PY{o}{.}\PY{n}{model}\PY{o}{.}\PY{n}{geo}\PY{o}{.}\PY{n}{addLine}\PY{p}{(}\PY{n}{p4}\PY{p}{,} \PY{n}{p1}\PY{p}{)}

    \PY{c+c1}{\PYZsh{} Создаем замкнутый контур прямоугольника}
    \PY{n}{cl} \PY{o}{=} \PY{n}{gmsh}\PY{o}{.}\PY{n}{model}\PY{o}{.}\PY{n}{geo}\PY{o}{.}\PY{n}{addCurveLoop}\PY{p}{(}\PY{p}{[}\PY{n}{l1}\PY{p}{,} \PY{n}{l2}\PY{p}{,} \PY{n}{l3}\PY{p}{,} \PY{n}{l4}\PY{p}{]}\PY{p}{)}
    \PY{n}{surface} \PY{o}{=} \PY{n}{gmsh}\PY{o}{.}\PY{n}{model}\PY{o}{.}\PY{n}{geo}\PY{o}{.}\PY{n}{addPlaneSurface}\PY{p}{(}\PY{p}{[}\PY{n}{cl}\PY{p}{]}\PY{p}{)}
    \PY{n}{gmsh}\PY{o}{.}\PY{n}{model}\PY{o}{.}\PY{n}{addPhysicalGroup}\PY{p}{(}\PY{l+m+mi}{2}\PY{p}{,} \PY{p}{[}\PY{n}{surface}\PY{p}{]}\PY{p}{,} \PY{l+m+mi}{1}\PY{p}{)}
    \PY{n}{gmsh}\PY{o}{.}\PY{n}{model}\PY{o}{.}\PY{n}{addPhysicalGroup}\PY{p}{(}\PY{l+m+mi}{1}\PY{p}{,} \PY{p}{[}\PY{n}{l4}\PY{p}{,} \PY{n}{l1}\PY{p}{,} \PY{n}{l2}\PY{p}{,} \PY{n}{l3}\PY{p}{]}\PY{p}{,} \PY{l+m+mi}{1}\PY{p}{)}
    \PY{n}{gmsh}\PY{o}{.}\PY{n}{model}\PY{o}{.}\PY{n}{geo}\PY{o}{.}\PY{n}{synchronize}\PY{p}{(}\PY{p}{)}
    \PY{n}{gmsh}\PY{o}{.}\PY{n}{model}\PY{o}{.}\PY{n}{mesh}\PY{o}{.}\PY{n}{generate}\PY{p}{(}\PY{l+m+mi}{2}\PY{p}{)}
    \PY{n}{gmsh}\PY{o}{.}\PY{n}{option}\PY{o}{.}\PY{n}{setNumber}\PY{p}{(}\PY{l+s+s2}{\PYZdq{}}\PY{l+s+s2}{Mesh.MshFileVersion}\PY{l+s+s2}{\PYZdq{}}\PY{p}{,} \PY{l+m+mi}{2}\PY{p}{)}
    \PY{n}{gmsh}\PY{o}{.}\PY{n}{write}\PY{p}{(}\PY{n}{name}\PY{p}{)}
    \PY{n}{gmsh}\PY{o}{.}\PY{n}{finalize}\PY{p}{(}\PY{p}{)}

\PY{n}{create\PYZus{}model}\PY{p}{(}\PY{l+m+mi}{1}\PY{p}{,} \PY{l+m+mi}{1}\PY{p}{,} \PY{l+m+mf}{0.1}\PY{p}{,} \PY{l+s+s1}{\PYZsq{}}\PY{l+s+s1}{./models/m1.msh}\PY{l+s+s1}{\PYZsq{}}\PY{p}{)}
\PY{n}{create\PYZus{}model}\PY{p}{(}\PY{l+m+mi}{1}\PY{p}{,} \PY{l+m+mi}{1}\PY{p}{,} \PY{l+m+mf}{0.05}\PY{p}{,} \PY{l+s+s1}{\PYZsq{}}\PY{l+s+s1}{./models/m2.msh}\PY{l+s+s1}{\PYZsq{}}\PY{p}{)}
\PY{n}{create\PYZus{}model}\PY{p}{(}\PY{l+m+mi}{1}\PY{p}{,} \PY{l+m+mi}{1}\PY{p}{,} \PY{l+m+mf}{0.01}\PY{p}{,} \PY{l+s+s1}{\PYZsq{}}\PY{l+s+s1}{./models/correct.msh}\PY{l+s+s1}{\PYZsq{}}\PY{p}{)}
\end{Verbatim}
\end{tcolorbox}
    \begin{Verbatim}[commandchars=\\\{\}]
Info    : Meshing 1D{\ldots}
Info    : [  0\%] Meshing curve 1 (Line)
Info    : [ 30\%] Meshing curve 2 (Line)
Info    : [ 50\%] Meshing curve 3 (Line)
Info    : [ 80\%] Meshing curve 4 (Line)
Info    : Done meshing 1D (Wall 0.000563026s, CPU 0.000938s)
Info    : Meshing 2D{\ldots}
Info    : Meshing surface 1 (Plane, Frontal-Delaunay)
Info    : Done meshing 2D (Wall 0.00593059s, CPU 0.005538s)
Info    : 142 nodes 286 elements
Info    : Writing './models/m1.msh'{\ldots}
Info    : Done writing './models/m1.msh'
Info    : Meshing 1D{\ldots}
Info    : [  0\%] Meshing curve 1 (Line)
Info    : [ 30\%] Meshing curve 2 (Line)
Info    : [ 50\%] Meshing curve 3 (Line)
Info    : [ 80\%] Meshing curve 4 (Line)
Info    : Done meshing 1D (Wall 0.000560197s, CPU 0.000832s)
Info    : Meshing 2D{\ldots}
Info    : Meshing surface 1 (Plane, Frontal-Delaunay)
Info    : Done meshing 2D (Wall 0.022705s, CPU 0.022764s)
Info    : 513 nodes 1028 elements
Info    : Writing './models/m2.msh'{\ldots}
Info    : Done writing './models/m2.msh'
Info    : Meshing 1D{\ldots}
Info    : [  0\%] Meshing curve 1 (Line)
Info    : [ 30\%] Meshing curve 2 (Line)
Info    : [ 50\%] Meshing curve 3 (Line)
Info    : [ 80\%] Meshing curve 4 (Line)
Info    : Done meshing 1D (Wall 0.000763172s, CPU 0.00083s)
Info    : Meshing 2D{\ldots}
Info    : Meshing surface 1 (Plane, Frontal-Delaunay)
Info    : Done meshing 2D (Wall 0.323248s, CPU 0.323338s)
Info    : 11833 nodes 23668 elements
Info    : Writing './models/correct.msh'{\ldots}
Info    : Done writing './models/correct.msh'
    \end{Verbatim}

    \begin{tcolorbox}[breakable, size=fbox, boxrule=1pt, pad at break*=1mm,colback=cellbackground, colframe=cellborder]
\prompt{In}{incolor}{47}{\boxspacing}
\begin{Verbatim}[commandchars=\\\{\}]
\PY{n}{show}\PY{p}{(}\PY{l+s+s1}{\PYZsq{}}\PY{l+s+s1}{./models/correct.msh}\PY{l+s+s1}{\PYZsq{}}\PY{p}{)}
\end{Verbatim}
\end{tcolorbox}

    \begin{Verbatim}[commandchars=\\\{\}]

    \end{Verbatim}

    \begin{center}
    \adjustimage{max size={0.9\linewidth}{0.9\paperheight}}{output_3_1.png}
    \end{center}
    { \hspace*{\fill} \\}
    
    \begin{tcolorbox}[breakable, size=fbox, boxrule=1pt, pad at break*=1mm,colback=cellbackground, colframe=cellborder]
\prompt{In}{incolor}{48}{\boxspacing}
\begin{Verbatim}[commandchars=\\\{\}]
\PY{n}{show}\PY{p}{(}\PY{l+s+s1}{\PYZsq{}}\PY{l+s+s1}{./models/m1.msh}\PY{l+s+s1}{\PYZsq{}}\PY{p}{)}
\end{Verbatim}
\end{tcolorbox}

    \begin{Verbatim}[commandchars=\\\{\}]

    \end{Verbatim}

    \begin{center}
    \adjustimage{max size={0.9\linewidth}{0.9\paperheight}}{output_4_1.png}
    \end{center}
    { \hspace*{\fill} \\}
    
    \begin{tcolorbox}[breakable, size=fbox, boxrule=1pt, pad at break*=1mm,colback=cellbackground, colframe=cellborder]
\prompt{In}{incolor}{49}{\boxspacing}
\begin{Verbatim}[commandchars=\\\{\}]
\PY{n}{show}\PY{p}{(}\PY{l+s+s1}{\PYZsq{}}\PY{l+s+s1}{./models/m2.msh}\PY{l+s+s1}{\PYZsq{}}\PY{p}{)}
\end{Verbatim}
\end{tcolorbox}

    \begin{Verbatim}[commandchars=\\\{\}]

    \end{Verbatim}

    \begin{center}
    \adjustimage{max size={0.9\linewidth}{0.9\paperheight}}{output_5_1.png}
    \end{center}
    { \hspace*{\fill} \\}
    

\Large\textbf{Слабая формула}\\
$$V=\{v\in H^1(\Omega):v=0, x \in \partial\Omega\}$$
Ищутся $v\in W$ такой, что \\
$$a(u,v)=L(v)$$
для всех $v\in W,$ где билинейные формы и линейные формы задаются следующим образом:\\
\begin{gather*}
a(u,v)=\int_{\Omega}\frac{u}{\tau}vdx+\int_{\Omega}\lambda\nabla{u}\cdot\nabla{v}dx \\
L(v)=\int_{\Omega}\frac{\check{u}}{\tau}vdx + \int_{\Omega}fvdx
\end{gather*}
\Large\textbf{{Реализация на Fenics 2019.2.0.dev0}}
    \begin{tcolorbox}[breakable, size=fbox, boxrule=1pt, pad at break*=1mm,colback=cellbackground, colframe=cellborder]
\prompt{In}{incolor}{6}{\boxspacing}
\begin{Verbatim}[commandchars=\\\{\}]
\PY{k+kn}{from} \PY{n+nn}{dolfin} \PY{k+kn}{import} \PY{o}{*}
\end{Verbatim}
\end{tcolorbox}

    \begin{tcolorbox}[breakable, size=fbox, boxrule=1pt, pad at break*=1mm,colback=cellbackground, colframe=cellborder]
\prompt{In}{incolor}{7}{\boxspacing}
\begin{Verbatim}[commandchars=\\\{\}]
\PY{k}{class} \PY{n+nc}{Solver}\PY{p}{(}\PY{p}{)}\PY{p}{:}
    \PY{k}{def} \PY{n+nf+fm}{\PYZus{}\PYZus{}init\PYZus{}\PYZus{}}\PY{p}{(}\PY{n+nb+bp}{self}\PY{p}{,} \PY{n}{name}\PY{p}{:} \PY{n+nb}{str}\PY{p}{)}\PY{p}{:}
        \PY{n+nb+bp}{self}\PY{o}{.}\PY{n}{\PYZus{}\PYZus{}loadMesh}\PY{p}{(}\PY{n}{name}\PY{p}{)}
        \PY{n+nb+bp}{self}\PY{o}{.}\PY{n}{bcs} \PY{o}{=} \PY{p}{[}\PY{p}{]}
        \PY{n+nb+bp}{self}\PY{o}{.}\PY{n}{u} \PY{o}{=} \PY{n}{TrialFunction}\PY{p}{(}\PY{n+nb+bp}{self}\PY{o}{.}\PY{n}{V}\PY{p}{)}
        \PY{n+nb+bp}{self}\PY{o}{.}\PY{n}{v} \PY{o}{=} \PY{n}{TestFunction}\PY{p}{(}\PY{n+nb+bp}{self}\PY{o}{.}\PY{n}{V}\PY{p}{)}
        
       
        \PY{n+nb+bp}{self}\PY{o}{.}\PY{n}{T} \PY{o}{=} \PY{l+m+mf}{1.0}
        \PY{n+nb+bp}{self}\PY{o}{.}\PY{n}{dt} \PY{o}{=} \PY{n+nb+bp}{self}\PY{o}{.}\PY{n}{T} \PY{o}{/} \PY{l+m+mi}{10}
        \PY{n+nb+bp}{self}\PY{o}{.}\PY{n}{lamda} \PY{o}{=} \PY{n}{Constant}\PY{p}{(}\PY{l+m+mf}{1.0}\PY{p}{)}
        \PY{n+nb+bp}{self}\PY{o}{.}\PY{n}{C} \PY{o}{=} \PY{n}{Constant}\PY{p}{(}\PY{l+m+mf}{1.0}\PY{p}{)}
        \PY{n+nb+bp}{self}\PY{o}{.}\PY{n}{f} \PY{o}{=} \PY{n}{Constant}\PY{p}{(}\PY{l+m+mf}{0.0}\PY{p}{)}

        \PY{n+nb+bp}{self}\PY{o}{.}\PY{n}{\PYZus{}\PYZus{}loadBoundaries}\PY{p}{(}\PY{p}{)}
  

        \PY{n}{u0\PYZus{}val} \PY{o}{=} \PY{n}{Constant}\PY{p}{(}\PY{l+m+mi}{20}\PY{p}{)}
        \PY{n+nb+bp}{self}\PY{o}{.}\PY{n}{u0} \PY{o}{=} \PY{n}{interpolate}\PY{p}{(}\PY{n}{u0\PYZus{}val}\PY{p}{,} \PY{n+nb+bp}{self}\PY{o}{.}\PY{n}{V}\PY{p}{)}

        \PY{n+nb+bp}{self}\PY{o}{.}\PY{n}{\PYZus{}\PYZus{}loadVelocity}\PY{p}{(}\PY{p}{)}
    \PY{k}{def} \PY{n+nf}{\PYZus{}\PYZus{}loadMesh}\PY{p}{(}\PY{n+nb+bp}{self}\PY{p}{,} \PY{n}{name}\PY{p}{:} \PY{n+nb}{str}\PY{p}{)}\PY{p}{:}
        \PY{n+nb+bp}{self}\PY{o}{.}\PY{n}{mesh} \PY{o}{=} \PY{n}{Mesh}\PY{p}{(}\PY{l+s+sa}{f}\PY{l+s+s1}{\PYZsq{}}\PY{l+s+si}{\PYZob{}name\PYZcb{}}\PY{l+s+s1}{.xml}\PY{l+s+s1}{\PYZsq{}}\PY{p}{)}
        \PY{n+nb+bp}{self}\PY{o}{.}\PY{n}{subdomains} \PY{o}{=} \PY{n}{MeshFunction}\PY{p}{(}\PY{l+s+s1}{\PYZsq{}}\PY{l+s+s1}{size\PYZus{}t}\PY{l+s+s1}{\PYZsq{}}\PY{p}{,} \PY{n+nb+bp}{self}\PY{o}{.}\PY{n}{mesh}\PY{p}{,} \PY{l+s+sa}{f}\PY{l+s+s1}{\PYZsq{}}\PY{l+s+si}{\PYZob{}name\PYZcb{}}\PY{l+s+s1}{\PYZus{}physical\PYZus{}region.xml}\PY{l+s+s1}{\PYZsq{}}\PY{p}{)}
        \PY{n+nb+bp}{self}\PY{o}{.}\PY{n}{boundaries} \PY{o}{=} \PY{n}{MeshFunction}\PY{p}{(}\PY{l+s+s1}{\PYZsq{}}\PY{l+s+s1}{size\PYZus{}t}\PY{l+s+s1}{\PYZsq{}}\PY{p}{,} \PY{n+nb+bp}{self}\PY{o}{.}\PY{n}{mesh}\PY{p}{,} \PY{l+s+sa}{f}\PY{l+s+s1}{\PYZsq{}}\PY{l+s+si}{\PYZob{}name\PYZcb{}}\PY{l+s+s1}{\PYZus{}facet\PYZus{}region.xml}\PY{l+s+s1}{\PYZsq{}}\PY{p}{)}
        \PY{n+nb+bp}{self}\PY{o}{.}\PY{n}{V} \PY{o}{=} \PY{n}{FunctionSpace}\PY{p}{(}\PY{n+nb+bp}{self}\PY{o}{.}\PY{n}{mesh}\PY{p}{,} \PY{l+s+s2}{\PYZdq{}}\PY{l+s+s2}{CG}\PY{l+s+s2}{\PYZdq{}}\PY{p}{,} \PY{l+m+mi}{1}\PY{p}{)}
        
        \PY{n+nb+bp}{self}\PY{o}{.}\PY{n}{dx} \PY{o}{=} \PY{n}{Measure}\PY{p}{(}\PY{l+s+s1}{\PYZsq{}}\PY{l+s+s1}{dx}\PY{l+s+s1}{\PYZsq{}}\PY{p}{,} \PY{n}{domain} \PY{o}{=} \PY{n+nb+bp}{self}\PY{o}{.}\PY{n}{mesh}\PY{p}{,} \PY{n}{subdomain\PYZus{}data} \PY{o}{=} \PY{n+nb+bp}{self}\PY{o}{.}\PY{n}{subdomains}\PY{p}{)}
        \PY{n+nb+bp}{self}\PY{o}{.}\PY{n}{ds} \PY{o}{=} \PY{n}{Measure}\PY{p}{(}\PY{l+s+s1}{\PYZsq{}}\PY{l+s+s1}{ds}\PY{l+s+s1}{\PYZsq{}}\PY{p}{,} \PY{n}{domain} \PY{o}{=} \PY{n+nb+bp}{self}\PY{o}{.}\PY{n}{mesh}\PY{p}{,} \PY{n}{subdomain\PYZus{}data} \PY{o}{=} \PY{n+nb+bp}{self}\PY{o}{.}\PY{n}{boundaries}\PY{p}{)}
    \PY{k}{def} \PY{n+nf}{\PYZus{}\PYZus{}loadBoundaries}\PY{p}{(}\PY{n+nb+bp}{self}\PY{p}{)}\PY{p}{:}
        \PY{n}{T1} \PY{o}{=} \PY{n}{Constant}\PY{p}{(}\PY{l+m+mf}{0.0}\PY{p}{)}
        \PY{n}{bc\PYZus{}1} \PY{o}{=} \PY{n}{DirichletBC}\PY{p}{(}\PY{n+nb+bp}{self}\PY{o}{.}\PY{n}{V}\PY{p}{,} \PY{n}{T1}\PY{p}{,} \PY{n+nb+bp}{self}\PY{o}{.}\PY{n}{boundaries}\PY{p}{,} \PY{l+m+mi}{1}\PY{p}{)}
        \PY{n+nb+bp}{self}\PY{o}{.}\PY{n}{bcs} \PY{o}{=} \PY{p}{[}\PY{n}{bc\PYZus{}1}\PY{p}{]}
    \PY{k}{def} \PY{n+nf}{\PYZus{}\PYZus{}loadVelocity}\PY{p}{(}\PY{n+nb+bp}{self}\PY{p}{)}\PY{p}{:}
        \PY{n+nb+bp}{self}\PY{o}{.}\PY{n}{a} \PY{o}{=} \PY{p}{(}\PY{n+nb+bp}{self}\PY{o}{.}\PY{n}{C}\PY{o}{/}\PY{n+nb+bp}{self}\PY{o}{.}\PY{n}{dt}\PY{p}{)} \PY{o}{*} \PY{n+nb+bp}{self}\PY{o}{.}\PY{n}{u} \PY{o}{*} \PY{n+nb+bp}{self}\PY{o}{.}\PY{n}{v} \PY{o}{*} \PY{n+nb+bp}{self}\PY{o}{.}\PY{n}{dx} \PY{o}{+} \PY{n+nb+bp}{self}\PY{o}{.}\PY{n}{lamda} \PY{o}{*} \PY{n}{inner}\PY{p}{(}\PY{n}{grad}\PY{p}{(}\PY{n+nb+bp}{self}\PY{o}{.}\PY{n}{u}\PY{p}{)}\PY{p}{,} \PY{n}{grad}\PY{p}{(}\PY{n+nb+bp}{self}\PY{o}{.}\PY{n}{v}\PY{p}{)}\PY{p}{)} \PY{o}{*} \PY{n+nb+bp}{self}\PY{o}{.}\PY{n}{dx}
        \PY{n+nb+bp}{self}\PY{o}{.}\PY{n}{L} \PY{o}{=} \PY{p}{(}\PY{n+nb+bp}{self}\PY{o}{.}\PY{n}{C}\PY{o}{/}\PY{n+nb+bp}{self}\PY{o}{.}\PY{n}{dt}\PY{p}{)} \PY{o}{*} \PY{n+nb+bp}{self}\PY{o}{.}\PY{n}{u0} \PY{o}{*} \PY{n+nb+bp}{self}\PY{o}{.}\PY{n}{v} \PY{o}{*} \PY{n+nb+bp}{self}\PY{o}{.}\PY{n}{dx} \PY{o}{+} \PY{n+nb+bp}{self}\PY{o}{.}\PY{n}{f} \PY{o}{*} \PY{n+nb+bp}{self}\PY{o}{.}\PY{n}{v} \PY{o}{*} \PY{n+nb+bp}{self}\PY{o}{.}\PY{n}{dx}
    \PY{k}{def} \PY{n+nf}{solve}\PY{p}{(}\PY{n+nb+bp}{self}\PY{p}{)}\PY{p}{:}
        \PY{n}{solutions} \PY{o}{=} \PY{p}{[}\PY{p}{]}
        \PY{n}{t} \PY{o}{=} \PY{l+m+mi}{0}
        \PY{n}{u} \PY{o}{=} \PY{n}{Function}\PY{p}{(}\PY{n+nb+bp}{self}\PY{o}{.}\PY{n}{V}\PY{p}{)}
        \PY{k}{while} \PY{n}{t} \PY{o}{\PYZlt{}} \PY{n+nb+bp}{self}\PY{o}{.}\PY{n}{T} \PY{o}{+} \PY{n}{DOLFIN\PYZus{}EPS}\PY{p}{:}
            \PY{n}{t} \PY{o}{+}\PY{o}{=} \PY{n+nb+bp}{self}\PY{o}{.}\PY{n}{dt}
            \PY{n}{solve}\PY{p}{(}\PY{n+nb+bp}{self}\PY{o}{.}\PY{n}{a} \PY{o}{==} \PY{n+nb+bp}{self}\PY{o}{.}\PY{n}{L}\PY{p}{,} \PY{n}{u}\PY{p}{,} \PY{n+nb+bp}{self}\PY{o}{.}\PY{n}{bcs}\PY{p}{)}
            \PY{n+nb+bp}{self}\PY{o}{.}\PY{n}{u0}\PY{o}{.}\PY{n}{assign}\PY{p}{(}\PY{n}{u}\PY{p}{)}
            \PY{n}{solutions}\PY{o}{.}\PY{n}{append}\PY{p}{(}\PY{n}{u}\PY{o}{.}\PY{n}{copy}\PY{p}{(}\PY{n}{deepcopy}\PY{o}{=}\PY{k+kc}{True}\PY{p}{)}\PY{p}{)}
        \PY{k}{return} \PY{n}{solutions}
\end{Verbatim}
\end{tcolorbox}

    \begin{tcolorbox}[breakable, size=fbox, boxrule=1pt, pad at break*=1mm,colback=cellbackground, colframe=cellborder]
\prompt{In}{incolor}{8}{\boxspacing}
\begin{Verbatim}[commandchars=\\\{\}]
\PY{n}{solver} \PY{o}{=} \PY{n}{Solver}\PY{p}{(}\PY{l+s+s1}{\PYZsq{}}\PY{l+s+s1}{./models/correct}\PY{l+s+s1}{\PYZsq{}}\PY{p}{)}
\end{Verbatim}
\end{tcolorbox}

    \begin{tcolorbox}[breakable, size=fbox, boxrule=1pt, pad at break*=1mm,colback=cellbackground, colframe=cellborder]
\prompt{In}{incolor}{9}{\boxspacing}
\begin{Verbatim}[commandchars=\\\{\}]
\PY{n}{solutions} \PY{o}{=} \PY{n}{solver}\PY{o}{.}\PY{n}{solve}\PY{p}{(}\PY{p}{)}
\end{Verbatim}
\end{tcolorbox}

    \begin{Verbatim}[commandchars=\\\{\}]
Calling FFC just-in-time (JIT) compiler, this may take some time.
Calling FFC just-in-time (JIT) compiler, this may take some time.
    \end{Verbatim}

    \begin{tcolorbox}[breakable, size=fbox, boxrule=1pt, pad at break*=1mm,colback=cellbackground, colframe=cellborder]
\prompt{In}{incolor}{10}{\boxspacing}
\begin{Verbatim}[commandchars=\\\{\}]
\PY{n}{uf} \PY{o}{=} \PY{n}{File}\PY{p}{(}\PY{l+s+s2}{\PYZdq{}}\PY{l+s+s2}{results/correct.pvd}\PY{l+s+s2}{\PYZdq{}}\PY{p}{)}
\PY{k}{for} \PY{n}{solve} \PY{o+ow}{in} \PY{n}{solutions}\PY{p}{:}
    \PY{n}{uf} \PY{o}{\PYZlt{}\PYZlt{}} \PY{n}{solve}
\end{Verbatim}
\end{tcolorbox}
    
    \begin{tcolorbox}[breakable, size=fbox, boxrule=1pt, pad at break*=1mm,colback=cellbackground, colframe=cellborder]
\prompt{In}{incolor}{8}{\boxspacing}
\begin{Verbatim}[commandchars=\\\{\}]
\PY{n}{solver1} \PY{o}{=} \PY{n}{Solver}\PY{p}{(}\PY{l+s+s1}{\PYZsq{}}\PY{l+s+s1}{./models/m1}\PY{l+s+s1}{\PYZsq{}}\PY{p}{)}
\PY{n}{solutions1} \PY{o}{=} \PY{n}{solver1}\PY{o}{.}\PY{n}{solve}\PY{p}{(}\PY{p}{)}
\PY{n}{solver2} \PY{o}{=} \PY{n}{Solver}\PY{p}{(}\PY{l+s+s1}{\PYZsq{}}\PY{l+s+s1}{./models/m2}\PY{l+s+s1}{\PYZsq{}}\PY{p}{)}
\PY{n}{solutions2} \PY{o}{=} \PY{n}{solver2}\PY{o}{.}\PY{n}{solve}\PY{p}{(}\PY{p}{)}
\end{Verbatim}
\end{tcolorbox}
\Large\textbf{Результаты}
    \begin{tcolorbox}[breakable, size=fbox, boxrule=1pt, pad at break*=1mm,colback=cellbackground, colframe=cellborder]
\prompt{In}{incolor}{15}{\boxspacing}
\begin{Verbatim}[commandchars=\\\{\}]
\PY{n}{errors} \PY{o}{=} \PY{p}{[}\PY{p}{]}
\PY{n}{errors1} \PY{o}{=} \PY{p}{[}\PY{p}{]}
\end{Verbatim}
\end{tcolorbox}

    \begin{tcolorbox}[breakable, size=fbox, boxrule=1pt, pad at break*=1mm,colback=cellbackground, colframe=cellborder]
\prompt{In}{incolor}{16}{\boxspacing}
\begin{Verbatim}[commandchars=\\\{\}]
\PY{k}{for} \PY{n}{i} \PY{o+ow}{in} \PY{n+nb}{range}\PY{p}{(}\PY{n+nb}{len}\PY{p}{(}\PY{n}{solutions}\PY{p}{)}\PY{p}{)}\PY{p}{:}
    \PY{n}{u\PYZus{}correct} \PY{o}{=} \PY{n}{solutions}\PY{p}{[}\PY{n}{i}\PY{p}{]}
    \PY{n}{u} \PY{o}{=} \PY{n}{solutions1}\PY{p}{[}\PY{n}{i}\PY{p}{]}
    \PY{n}{u\PYZus{}correct\PYZus{}another\PYZus{}V} \PY{o}{=} \PY{n}{interpolate}\PY{p}{(}\PY{n}{u\PYZus{}correct}\PY{p}{,} \PY{n}{solver}\PY{o}{.}\PY{n}{V}\PY{p}{)}
    \PY{n}{ABS} \PY{o}{=} \PY{n+nb}{abs}\PY{p}{(}\PY{n}{assemble}\PY{p}{(}\PY{p}{(}\PY{n}{u\PYZus{}correct\PYZus{}another\PYZus{}V} \PY{o}{\PYZhy{}} \PY{n}{u}\PY{p}{)}\PY{o}{*}\PY{n}{solver1}\PY{o}{.}\PY{n}{dx}\PY{p}{)}\PY{p}{)}\PY{o}{*}\PY{l+m+mi}{100}
    \PY{n}{t} \PY{o}{=} \PY{n}{inner}\PY{p}{(}\PY{n}{u} \PY{o}{\PYZhy{}} \PY{n}{u\PYZus{}correct\PYZus{}another\PYZus{}V}\PY{p}{,} \PY{n}{u} \PY{o}{\PYZhy{}} \PY{n}{u\PYZus{}correct\PYZus{}another\PYZus{}V}\PY{p}{)}\PY{o}{*}\PY{n}{solver1}\PY{o}{.}\PY{n}{dx}
    \PY{n}{b} \PY{o}{=} \PY{n}{inner}\PY{p}{(}\PY{n}{u}\PY{p}{,}\PY{n}{u}\PY{p}{)}\PY{o}{*}\PY{n}{solver1}\PY{o}{.}\PY{n}{dx}
    \PY{n}{L2} \PY{o}{=} \PY{n}{sqrt}\PY{p}{(}\PY{n+nb}{abs}\PY{p}{(}\PY{n}{assemble}\PY{p}{(}\PY{n}{t}\PY{p}{)}\PY{p}{)}\PY{o}{/}\PY{n+nb}{abs}\PY{p}{(}\PY{n}{assemble}\PY{p}{(}\PY{n}{b}\PY{p}{)}\PY{p}{)}\PY{p}{)}\PY{o}{*}\PY{l+m+mi}{100}

    \PY{n}{t} \PY{o}{=} \PY{n}{inner}\PY{p}{(}\PY{n}{grad}\PY{p}{(}\PY{n}{u} \PY{o}{\PYZhy{}} \PY{n}{u\PYZus{}correct\PYZus{}another\PYZus{}V}\PY{p}{)}\PY{p}{,} \PY{n}{grad}\PY{p}{(}\PY{n}{u} \PY{o}{\PYZhy{}} \PY{n}{u\PYZus{}correct\PYZus{}another\PYZus{}V}\PY{p}{)}\PY{p}{)}\PY{o}{*}\PY{n}{solver1}\PY{o}{.}\PY{n}{dx}
    \PY{n}{b} \PY{o}{=} \PY{n}{inner}\PY{p}{(}\PY{n}{grad}\PY{p}{(}\PY{n}{u}\PY{p}{)}\PY{p}{,}\PY{n}{grad}\PY{p}{(}\PY{n}{u}\PY{p}{)}\PY{p}{)}\PY{o}{*}\PY{n}{solver1}\PY{o}{.}\PY{n}{dx}
    \PY{n}{H1} \PY{o}{=} \PY{n}{sqrt}\PY{p}{(}\PY{n+nb}{abs}\PY{p}{(}\PY{n}{assemble}\PY{p}{(}\PY{n}{t}\PY{p}{)}\PY{p}{)}\PY{o}{/}\PY{n+nb}{abs}\PY{p}{(}\PY{n}{assemble}\PY{p}{(}\PY{n}{b}\PY{p}{)}\PY{p}{)}\PY{p}{)}\PY{o}{*}\PY{l+m+mi}{100}
    \PY{n}{errors}\PY{o}{.}\PY{n}{append}\PY{p}{(}\PY{p}{[}\PY{n}{ABS}\PY{p}{,} \PY{n}{L2}\PY{p}{,} \PY{n}{H1}\PY{p}{]}\PY{p}{)}
\end{Verbatim}
\end{tcolorbox}

    \begin{tcolorbox}[breakable, size=fbox, boxrule=1pt, pad at break*=1mm,colback=cellbackground, colframe=cellborder]
\prompt{In}{incolor}{17}{\boxspacing}
\begin{Verbatim}[commandchars=\\\{\}]
\PY{k}{for} \PY{n}{i} \PY{o+ow}{in} \PY{n+nb}{range}\PY{p}{(}\PY{n+nb}{len}\PY{p}{(}\PY{n}{solutions}\PY{p}{)}\PY{p}{)}\PY{p}{:}
    \PY{n}{u\PYZus{}correct} \PY{o}{=} \PY{n}{solutions}\PY{p}{[}\PY{n}{i}\PY{p}{]}
    \PY{n}{u} \PY{o}{=} \PY{n}{solutions2}\PY{p}{[}\PY{n}{i}\PY{p}{]}
    \PY{n}{u\PYZus{}correct\PYZus{}another\PYZus{}V} \PY{o}{=} \PY{n}{interpolate}\PY{p}{(}\PY{n}{u\PYZus{}correct}\PY{p}{,} \PY{n}{solver}\PY{o}{.}\PY{n}{V}\PY{p}{)}
    \PY{n}{ABS} \PY{o}{=} \PY{n+nb}{abs}\PY{p}{(}\PY{n}{assemble}\PY{p}{(}\PY{p}{(}\PY{n}{u\PYZus{}correct\PYZus{}another\PYZus{}V} \PY{o}{\PYZhy{}} \PY{n}{u}\PY{p}{)}\PY{o}{*}\PY{n}{solver1}\PY{o}{.}\PY{n}{dx}\PY{p}{)}\PY{p}{)}\PY{o}{*}\PY{l+m+mi}{100}
    \PY{n}{t} \PY{o}{=} \PY{n}{inner}\PY{p}{(}\PY{n}{u} \PY{o}{\PYZhy{}} \PY{n}{u\PYZus{}correct\PYZus{}another\PYZus{}V}\PY{p}{,} \PY{n}{u} \PY{o}{\PYZhy{}} \PY{n}{u\PYZus{}correct\PYZus{}another\PYZus{}V}\PY{p}{)}\PY{o}{*}\PY{n}{solver2}\PY{o}{.}\PY{n}{dx}
    \PY{n}{b} \PY{o}{=} \PY{n}{inner}\PY{p}{(}\PY{n}{u}\PY{p}{,}\PY{n}{u}\PY{p}{)}\PY{o}{*}\PY{n}{solver2}\PY{o}{.}\PY{n}{dx}
    \PY{n}{L2} \PY{o}{=} \PY{n}{sqrt}\PY{p}{(}\PY{n+nb}{abs}\PY{p}{(}\PY{n}{assemble}\PY{p}{(}\PY{n}{t}\PY{p}{)}\PY{p}{)}\PY{o}{/}\PY{n+nb}{abs}\PY{p}{(}\PY{n}{assemble}\PY{p}{(}\PY{n}{b}\PY{p}{)}\PY{p}{)}\PY{p}{)}\PY{o}{*}\PY{l+m+mi}{100}

    \PY{n}{t} \PY{o}{=} \PY{n}{inner}\PY{p}{(}\PY{n}{grad}\PY{p}{(}\PY{n}{u} \PY{o}{\PYZhy{}} \PY{n}{u\PYZus{}correct\PYZus{}another\PYZus{}V}\PY{p}{)}\PY{p}{,} \PY{n}{grad}\PY{p}{(}\PY{n}{u} \PY{o}{\PYZhy{}} \PY{n}{u\PYZus{}correct\PYZus{}another\PYZus{}V}\PY{p}{)}\PY{p}{)}\PY{o}{*}\PY{n}{solver2}\PY{o}{.}\PY{n}{dx}
    \PY{n}{b} \PY{o}{=} \PY{n}{inner}\PY{p}{(}\PY{n}{grad}\PY{p}{(}\PY{n}{u}\PY{p}{)}\PY{p}{,}\PY{n}{grad}\PY{p}{(}\PY{n}{u}\PY{p}{)}\PY{p}{)}\PY{o}{*}\PY{n}{solver2}\PY{o}{.}\PY{n}{dx}
    \PY{n}{H1} \PY{o}{=} \PY{n}{sqrt}\PY{p}{(}\PY{n+nb}{abs}\PY{p}{(}\PY{n}{assemble}\PY{p}{(}\PY{n}{t}\PY{p}{)}\PY{p}{)}\PY{o}{/}\PY{n+nb}{abs}\PY{p}{(}\PY{n}{assemble}\PY{p}{(}\PY{n}{b}\PY{p}{)}\PY{p}{)}\PY{p}{)}\PY{o}{*}\PY{l+m+mi}{100}
    \PY{n}{errors1}\PY{o}{.}\PY{n}{append}\PY{p}{(}\PY{p}{[}\PY{n}{ABS}\PY{p}{,} \PY{n}{L2}\PY{p}{,} \PY{n}{H1}\PY{p}{]}\PY{p}{)}
\end{Verbatim}
\end{tcolorbox}

    \begin{Verbatim}[commandchars=\\\{\}]
Calling FFC just-in-time (JIT) compiler, this may take some time.
    \end{Verbatim}

    \begin{tcolorbox}[breakable, size=fbox, boxrule=1pt, pad at break*=1mm,colback=cellbackground, colframe=cellborder]
\prompt{In}{incolor}{28}{\boxspacing}
\begin{Verbatim}[commandchars=\\\{\}]
\PY{k+kn}{import} \PY{n+nn}{matplotlib}\PY{n+nn}{.}\PY{n+nn}{pyplot} \PY{k}{as} \PY{n+nn}{plt}
\end{Verbatim}
\end{tcolorbox}

    \begin{tcolorbox}[breakable, size=fbox, boxrule=1pt, pad at break*=1mm,colback=cellbackground, colframe=cellborder]
\prompt{In}{incolor}{29}{\boxspacing}
\begin{Verbatim}[commandchars=\\\{\}]
\PY{n}{xerr} \PY{o}{=} \PY{p}{[}\PY{n}{i} \PY{k}{for} \PY{n}{i} \PY{o+ow}{in} \PY{n+nb}{range}\PY{p}{(}\PY{n+nb}{len}\PY{p}{(}\PY{n}{errors}\PY{p}{)}\PY{p}{)}\PY{p}{]}
\PY{n}{yerr} \PY{o}{=} \PY{p}{[}\PY{n}{r}\PY{p}{[}\PY{l+m+mi}{0}\PY{p}{]} \PY{k}{for} \PY{n}{r} \PY{o+ow}{in} \PY{n}{errors}\PY{p}{]}
\PY{n}{yerr1} \PY{o}{=} \PY{p}{[}\PY{n}{r}\PY{p}{[}\PY{l+m+mi}{0}\PY{p}{]} \PY{k}{for} \PY{n}{r} \PY{o+ow}{in} \PY{n}{errors1}\PY{p}{]}
\PY{n}{plt}\PY{o}{.}\PY{n}{figure}\PY{p}{(}\PY{n}{figsize}\PY{o}{=}\PY{p}{(}\PY{l+m+mi}{8}\PY{p}{,} \PY{l+m+mi}{8}\PY{p}{)}\PY{p}{)}
\PY{n}{plt}\PY{o}{.}\PY{n}{plot}\PY{p}{(}\PY{n}{xerr}\PY{p}{,} \PY{n}{yerr}\PY{p}{,} \PY{n}{label}\PY{o}{=}\PY{l+s+s2}{\PYZdq{}}\PY{l+s+s2}{m1}\PY{l+s+s2}{\PYZdq{}}\PY{p}{)}
\PY{n}{plt}\PY{o}{.}\PY{n}{plot}\PY{p}{(}\PY{n}{xerr}\PY{p}{,} \PY{n}{yerr1}\PY{p}{,} \PY{n}{label}\PY{o}{=}\PY{l+s+s2}{\PYZdq{}}\PY{l+s+s2}{m2}\PY{l+s+s2}{\PYZdq{}}\PY{p}{)}
\PY{n}{plt}\PY{o}{.}\PY{n}{legend}\PY{p}{(}\PY{p}{)}
\PY{n}{plt}\PY{o}{.}\PY{n}{show}\PY{p}{(}\PY{p}{)}
\end{Verbatim}
\end{tcolorbox}
\centering{Абсолютная ошибка}
    \adjustimage{max size={0.9\linewidth}{0.9\paperheight},center}{output_20_0.png}\\ 
    { \hspace*{\fill} \\}
    
    \begin{tcolorbox}[breakable, size=fbox, boxrule=1pt, pad at break*=1mm,colback=cellbackground, colframe=cellborder]
\prompt{In}{incolor}{25}{\boxspacing}
\begin{Verbatim}[commandchars=\\\{\}]
\PY{n}{xerr} \PY{o}{=} \PY{p}{[}\PY{n}{i} \PY{k}{for} \PY{n}{i} \PY{o+ow}{in} \PY{n+nb}{range}\PY{p}{(}\PY{n+nb}{len}\PY{p}{(}\PY{n}{errors}\PY{p}{)}\PY{p}{)}\PY{p}{]}
\PY{n}{yerr} \PY{o}{=} \PY{p}{[}\PY{n}{r}\PY{p}{[}\PY{l+m+mi}{1}\PY{p}{]} \PY{k}{for} \PY{n}{r} \PY{o+ow}{in} \PY{n}{errors}\PY{p}{]}
\PY{n}{yerr1} \PY{o}{=} \PY{p}{[}\PY{n}{r}\PY{p}{[}\PY{l+m+mi}{1}\PY{p}{]} \PY{k}{for} \PY{n}{r} \PY{o+ow}{in} \PY{n}{errors1}\PY{p}{]}
\PY{n}{plt}\PY{o}{.}\PY{n}{figure}\PY{p}{(}\PY{n}{figsize}\PY{o}{=}\PY{p}{(}\PY{l+m+mi}{8}\PY{p}{,} \PY{l+m+mi}{8}\PY{p}{)}\PY{p}{)}
\PY{n}{plt}\PY{o}{.}\PY{n}{plot}\PY{p}{(}\PY{n}{xerr}\PY{p}{,} \PY{n}{yerr}\PY{p}{,} \PY{n}{label}\PY{o}{=}\PY{l+s+s2}{\PYZdq{}}\PY{l+s+s2}{m1}\PY{l+s+s2}{\PYZdq{}}\PY{p}{)}
\PY{n}{plt}\PY{o}{.}\PY{n}{plot}\PY{p}{(}\PY{n}{xerr}\PY{p}{,} \PY{n}{yerr1}\PY{p}{,} \PY{n}{label}\PY{o}{=}\PY{l+s+s2}{\PYZdq{}}\PY{l+s+s2}{m2}\PY{l+s+s2}{\PYZdq{}}\PY{p}{)}
\PY{n}{plt}\PY{o}{.}\PY{n}{legend}\PY{p}{(}\PY{p}{)}
\PY{n}{plt}\PY{o}{.}\PY{n}{show}\PY{p}{(}\PY{p}{)}
\end{Verbatim}
\end{tcolorbox}
L2 ошибка
    \begin{center}
    \adjustimage{max size={0.9\linewidth}{0.9\paperheight}}{output_21_0.png}
    \end{center}
    { \hspace*{\fill} \\}
    
    \begin{tcolorbox}[breakable, size=fbox, boxrule=1pt, pad at break*=1mm,colback=cellbackground, colframe=cellborder]
\prompt{In}{incolor}{27}{\boxspacing}
\begin{Verbatim}[commandchars=\\\{\}]
\PY{n}{xerr} \PY{o}{=} \PY{p}{[}\PY{n}{i} \PY{k}{for} \PY{n}{i} \PY{o+ow}{in} \PY{n+nb}{range}\PY{p}{(}\PY{n+nb}{len}\PY{p}{(}\PY{n}{errors}\PY{p}{)}\PY{p}{)}\PY{p}{]}
\PY{n}{yerr} \PY{o}{=} \PY{p}{[}\PY{n}{r}\PY{p}{[}\PY{l+m+mi}{2}\PY{p}{]} \PY{k}{for} \PY{n}{r} \PY{o+ow}{in} \PY{n}{errors}\PY{p}{]}
\PY{n}{yerr1} \PY{o}{=} \PY{p}{[}\PY{n}{r}\PY{p}{[}\PY{l+m+mi}{2}\PY{p}{]} \PY{k}{for} \PY{n}{r} \PY{o+ow}{in} \PY{n}{errors1}\PY{p}{]}
\PY{n}{plt}\PY{o}{.}\PY{n}{figure}\PY{p}{(}\PY{n}{figsize}\PY{o}{=}\PY{p}{(}\PY{l+m+mi}{8}\PY{p}{,} \PY{l+m+mi}{8}\PY{p}{)}\PY{p}{)}
\PY{n}{plt}\PY{o}{.}\PY{n}{plot}\PY{p}{(}\PY{n}{xerr}\PY{p}{,} \PY{n}{yerr}\PY{p}{,} \PY{n}{label}\PY{o}{=}\PY{l+s+s2}{\PYZdq{}}\PY{l+s+s2}{m1}\PY{l+s+s2}{\PYZdq{}}\PY{p}{)}
\PY{n}{plt}\PY{o}{.}\PY{n}{plot}\PY{p}{(}\PY{n}{xerr}\PY{p}{,} \PY{n}{yerr1}\PY{p}{,} \PY{n}{label}\PY{o}{=}\PY{l+s+s2}{\PYZdq{}}\PY{l+s+s2}{m2}\PY{l+s+s2}{\PYZdq{}}\PY{p}{)}
\PY{n}{plt}\PY{o}{.}\PY{n}{legend}\PY{p}{(}\PY{p}{)}
\PY{n}{plt}\PY{o}{.}\PY{n}{show}\PY{p}{(}\PY{p}{)}
\end{Verbatim}
\end{tcolorbox}
H1 ошибка
    \begin{center}
    \adjustimage{max size={0.9\linewidth}{0.9\paperheight}}{output_22_0.png}
    \end{center}
    { \hspace*{\fill} \\}
  
      \noindent\adjustimage{max size={0.9\linewidth}{0.9\paperheight},center,figure, label={text},caption={text2}}{correct.png}\centering{t=0}
      \noindent\adjustimage{max size={0.9\linewidth}{0.9\paperheight},center,figure, label={text},caption={text2}}{correct_middle.png}\centering{t=5}
      \noindent\adjustimage{max size={0.9\linewidth}{0.9\paperheight},center,figure, label={text},caption={text2}}{correct_last.png}\centering{t=10}


    % Add a bibliography block to the postdoc
    
    
    
\end{document}
